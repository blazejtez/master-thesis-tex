\noindent
\begingroup
\fontsize{12pt}{1.5pt}\selectfont
\textbf{ABSTRACT}
\endgroup

\vspace{3mm}

Ever since the introduction of the Compute Unified Device Architecture (CUDA), researchers have gained the capability to solve computationally challenging problems more efficiently, leading to faster results and greater numerical accuracy. This thesis contributes to this trend by introducing a novel approach for calculating hydrogen energy through Schr{\"o}dinger equation with finite difference method. By avoiding the need to explicitly store Hamiltonian operator matrix, the approach removes the $O(x^6)$ space complexity. The new solution was implemented and tested on the best high-performance customer graphics card available in 2024, the Nvidia RTX 4090.

Most of the project was developed in Python using the CuPy library, while some fragments were implemented in CUDA C. This work includes the implementation of several numerical algorithms for solving the Schrödinger equation eigenproblem through minimization of Rayleigh's quotient, as well as using existing LOBPCG algorithm implementation. The resulting wavefunctions were visualized using the mayavi Python module, with comparisons made to analytical solutions of hydrogen atom wavefunctions.

\textbf{Keywords:} finite difference method, Schrödinger equation, hydrogen atom.

\textbf{Field and science of technology in accordance with OECD requirements:}
physical science, computer and information sciences, computer sciences
