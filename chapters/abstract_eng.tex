\noindent
\begingroup
\fontsize{12pt}{1.5pt}\selectfont
\textbf{ABSTRACT}
\endgroup

\vspace{3mm}

Ever since the introduction of the Compute Unified Device Architecture (CUDA), researchers have gained the capability to solve computationally challenging problems more efficiently, leading to faster results and greater numerical accuracy. This thesis contributes to this trend by introducing a new method for calculating hydrogen energy through Schr{\"o}dinger equation with finite difference method. The new method was implemented and tested on the best currently available high-performance customer graphics card, the Nvidia RTX 4090.

Most of the project was developed in Python using the CuPy library, while some fragments were implemented in CUDA C.

This thesis demonstrates the implementation of several numerical algorithms for solving the Schrödinger equation eigenproblem and optimizes Rayleigh's quotient minimization. The resulting wavefunctions were then visualized using the mayavi Python module, with comparisons made to graphical representations of analytically derived wavefunctions.

\textbf{Keywords:} finite difference method, Schrödinger equation, hydrogen atom.

\textbf{Field and science of technology in accordance with OECD requirements:}
computer and information sciences, computer sciences
