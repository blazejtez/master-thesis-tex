\section{Introduction}

\subsection{Background}
% ok, czyli mówię, że inaczej się liczy niż kiedyś
% czy chcę o Kołosie jak przyjechał z działającym programem do USA...?

The landscape of scientific computation has substantially evolved since Schrödinger introduced his famous equation. Modern scientists possess tools that are far more advanced than pen and paper, enabling them to solve complex mathematical problems with accuracy and speed greater than ever. Numerical methods, once laborious and prone to human error due to iterative processes, have become indispensable in many STEM disciplines.

Graphics Processing Units (GPUs) have proven to be highly effective, powerful, high-throughput computing units. While Central Processing Units (CPUs) have been optimized for low-latency tasks, addressing more minor problems quickly, the scientific and high-performance computing communities have demanded greater throughput for large-scale data processing. This demand led to the Compute Unified Device Architecture (CUDA) development in 2006, which allowed GPUs to be harnessed as general-purpose computing devices.

GPUs' parallel processing capabilities align perfectly with numerical methods like FDM, which often involves computations on large grids. By distributing these calculations across thousands of cores, GPUs significantly reduce computation time, making them ideal for large-scale simulations in quantum mechanics.

According to Nvidia, the number of CUDA-enabled devices used worldwide surpassed 500 million -- and that information has yet to be updated in the last three years, which was the time of the company's exponential growth in sales. The wide range of applications included bioinformatics, computational chemistry, fluid dynamics, structural mechanics, data science, numerical analytics, and many others.\cite{cuda} 	

\subsection{Problem statement}

This study aims to develop a new efficient method for solving the Schrödinger Equation using the Finite Difference Method (FDM) on a GPU while reducing memory complexity. The traditional approach of directly representing the Hamiltonian matrix results in memory requirements that grow as $O(x\textsuperscript{6})$, making large-scale quantum simulations challenging to handle on modern GPUs. This research aims to express the Laplacian part of Hamiltonian indirectly through the FDM, significantly reducing the space complexity to $O(x\textsuperscript{3})$, where x represents the edge length of the Cartesian grid.
%% poniższy akapit do przedyskutowania
In addition to reducing memory complexity, this work will employ numerical methods to solve the eigenvalue problem in a search for the most suitable one. These methods include minimizing Rayleigh's quotient, both with and without orthogonal vector constraints, using a variety of algorithms, including gradient descent, perturbed gradient descent, LOBPCG, and ADAM.

Finally, the proposed method has to be validated. For that purpose, this study aims to calculate the hydrogen atom's energies and their respective eigenstates. The hydrogen atom has well-known analytical solutions, which is favorable for serving as a benchmark for evaluating the new method's accuracy. By comparing the results to analytical solutions, this work will demonstrate the validity of the approach in solving quantum mechanical problems.

\subsection{Significance of the study}

This study addresses a critical challenge in computational quantum mechanics: how to efficiently and accurately solve the Schrödinger equation. The method proposed in this work significantly reduces the memory complexity of solving the Schrödinger equation using the FDM. By lowering the complexity from $O(x\textsuperscript{6})$ to $O(x\textsuperscript{3})$, this study may lead to more extensive and intricate quantum systems to be simulated on GPUs. This is particularly relevant for the design of new materials, drug discovery, and quantum technologies, where accurate quantum models enable better predictions of drug properties, guide experimental work, and foster innovation.

\subsection{Thesis Structure}

The subsequent chapters of this thesis focus on the following aspects of the work:
\begin{itemize}[itemsep=0\baselineskip, topsep=1.5pt, parsep=1.5pt] 
	\item Chapter 2 describes selected existing methods for solving the Schrödinger equation.
	\item Chapter 3 outlines this study's theoretical framework and software tools. The chapter also highlights the integration of Python, CuPy, and CUDA to optimize numerical computations on GPUs.
	\item Chapter 4 presents the results delivered by the new method, including a comparative analysis of different tools and approaches.
	\item Chapter 5 offers conclusions based on the obtained results and proposes directions for future research.  
	\item The appendix contains detailed proofs derived manually during the development of this thesis. These proofs include mathematical derivations of Laplace operator stencils and optimization steps for numerical solvers.
\end{itemize}

