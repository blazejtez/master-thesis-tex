\section{Introduction}

\subsection{Background}

The landscape of scientific computation has substantially evolved since Schrödinger introduced his famous equation. Numerical methods, once laborious and prone to human error due to their iterative nature, have become indispensable in many STEM disciplines. Using computers as a scientific tool has enabled breakthroughs in virtually every field of study. In recent years, advancements in computational science have been driven by the emergence of Graphics Processing Units (GPUs) in high-performance computing.

GPUs have proven to be highly effective, powerful, high-throughput computing units. While Central Processing Units (CPUs) have been optimized for low-latency tasks, addressing more minor problems quickly, the scientific and high-performance computing communities have demanded greater throughput for large-scale data processing \cite{cheng2014professional}. This demand led to the Compute Unified Device Architecture (CUDA) development in 2006, which allowed GPUs to be harnessed as general-purpose computing devices \cite{cuda}

GPUs' parallel processing capabilities align perfectly with numerical methods like FDM, which often involves computations on large grids. By distributing these calculations across thousands of cores, GPUs significantly reduce computation time, making them ideal for large-scale simulations in quantum mechanics.

According to Nvidia, the number of CUDA-enabled devices used worldwide surpassed 500 million -- and that information has yet to be updated in the last three years, which was the time of the company's exponential growth in sales, both due to cryptocurrency mining and the following AI rush. The wide range of applications included bioinformatics, computational chemistry, fluid dynamics, structural mechanics, data science, numerical analytics, and many others \cite{cuda}.

\subsection{Problem statement}

This study aims to develop a new efficient method for solving the Schrödinger equation using the Finite Difference Method (FDM) on a GPU that reduces memory complexity. The traditional approach of directly representing the Hamiltonian matrix operator results in memory requirements that grow as $O(x\textsuperscript{6})$, where x represents the edge length of the considered cube in a Cartesian grid, making large-scale quantum simulations challenging to handle on modern GPUs. This research aims to express the Hamiltonian indirectly through the FDM by significantly reducing the space complexity as a result of overcoming this bottleneck.

In addition to reducing memory complexity, this work will employ numerical methods to solve the eigenvalue problem in a search for the one most suitable for calculating hydrogen eigenstates. These methods include minimizing Rayleigh's quotient using algorithms: gradient descent, LOBPCG, and Adam.

Finally, the proposed method has to be validated. For that purpose, this study aims to calculate the hydrogen atom's energies and their respective eigenstates. The hydrogen atom was proposed as a benchmark since it has well-known analytical solutions. By comparing the results to analytical solutions, this work will demonstrate the validity of the approach in solving quantum mechanical problems.

\subsection{Significance of the study}

This study addresses a critical challenge in computational quantum mechanics: how to efficiently and accurately solve the Schrödinger equation. The method proposed in this work significantly reduces the memory complexity of solving the Schrödinger equation using the FDM. By removing the space complexity bottleneck of $O(x\textsuperscript{6})$, this study may lead to more extensive and intricate quantum systems to be simulated on GPUs. This is particularly relevant for the design of nanotechnology, the science of new materials, drug discovery, and quantum technologies, where accurate quantum models enable better predictions of drug properties, guide experimental work, and foster innovation.

\subsection{Thesis Structure}

The subsequent chapters of this thesis focus on different aspects of the work as follows. Chapter 2 describes selected existing methods for solving the Schrödinger equation. Chapter 3 outlines this study's theoretical framework and software tools. The chapter also highlights the integration of Python, CuPy, and CUDA to optimize numerical computations on GPUs. Chapter 4 presents the results delivered by the new method, including a comparative analysis of different tools and approaches. Chapter 5 offers conclusions based on the results and proposes future research directions.
