\noindent
\begingroup
\fontsize{12pt}{1.5pt}\selectfont
\textbf{STRESZCZENIE}
\endgroup

\vspace{3mm}

Od czasu wprowadzenia Compute Unified Device Architecture (CUDA) naukowcy zyskali możliwość rozwiązywania obliczeniowo trudnych problemów w bardziej efektywny sposób, co prowadzi do szybszych wyników i większej dokładności numerycznej. Niniejsza praca wpisuje się w ten trend, przedstawiając nowatorskie podejście do obliczania energii wodoru za pomocą równania Schrödingera i metody różnic skończonych. Dzięki uniknięciu konieczności explicite przechowywania macierzy operatora Hamiltona to podejście eliminuje złożoność pamięciową $O(x^6)$. Rozwiązanie zostało zaimplementowane i przetestowane na najlepszej konsumenckiej karcie graficznej dostępnej w 2024 roku, Nvidia RTX 4090.

Większość projektu została opracowana w języku Python z wykorzystaniem biblioteki CuPy, podczas gdy niektóre fragmenty zostały zaimplementowane w CUDA C. Praca obejmuje implementację kilku algorytmów numerycznych do rozwiązywania problemu własnego równania Schrödingera poprzez minimalizację ilorazu Rayleigha, a także wykorzystanie istniejącej implementacji algorytmu LOBPCG. Uzyskane funkcje falowe zostały zwizualizowane przy użyciu modułu Python Mayavi, a wyniki porównano z graficznymi reprezentacjami analitycznie wyprowadzonych funkcji falowych.

\textbf{Słowa kluczowe:}: metoda różnic skończonych, równanie Schrödingera, atom wodoru.

\textbf{Dziedzina i dyscyplina naukowa zgodnie z wymaganiami OECD:}
nauki komputerowe i informatyczne, informatyka