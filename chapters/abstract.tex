\noindent
\begingroup
\fontsize{12pt}{1.5pt}\selectfont
\textbf{STRESZCZENIE}
\endgroup

\vspace{3mm}

Od momentu pojawienia się architektury Compute Unified Device Architecture (CUDA), badacze zyskali możliwość rozwiązywania w efektywniejszy sposób skomplikowanych obliczeniowo problemów, osiągając szybsze wyniki i większą dokładność numeryczną. Niniejszy projekt wpisuje się w ten trend, zawierając nową metodę obliczania stanu podstawowego i stanów wzbudzonych atomu wodoru z wykorzystaniem metody różnic skończonych. Implementację nowej metody i testy wykonano na konsumenckiej karcie graficznej o najwyższych możliwościach obliczeniowych w momencie publikacji, a mianowicie Nvidia RTX 4090.

Większość projektu została opracowana w języku Python z wykorzystaniem biblioteki CuPy, z wyłączeniem fragmentów zaimplementowanych wprost w CUDA C.

Ponadto w ramach tej pracy porównano algorytmy numeryczne służące do rozwiązywania problemu własnego równania Schr{\"o}dingera w celu optymalizacji minimalizacji ilorazu Rayleigha. Uzyskane funkcje falowe zostały zwizualizowane z wykorzystaniem mayavi - modułu z języka Python, a wyniki porównano z reprezentacjami graficznymi analitycznie wyprowadzonych funkcji falowych.

\textbf{Słowa kluczowe:}: metoda różnic skończonych, równanie Schrödingera, atom wodoru.

\textbf{Dziedzina i dyscyplina naukowa zgodnie z wymaganiami OECD:}
nauki komputerowe i informatyczne, informatyka