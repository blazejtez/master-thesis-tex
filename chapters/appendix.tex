\section{Appendices}
\subsection{Code}

The code developed during this master thesis is available upon request. For access contact me directly via email at blaz.tez@gmail.com. The associated GitHub repository URL is \url{https://github.com/blazejtez/qc}.

\subsection{Detailed proofs}
\subsubsection{7-point stencil}
Derivation begins with the Laplace operator:
\begin{equation}
	\Delta = \frac{\partial^2}{\partial x^2} + \frac{\partial^2}{\partial y^2} + \frac{\partial^2}{\partial z^2}
	\label{eq1}
\end{equation}
The derivative can be approximated as a difference quotient of:
\begin{equation}
	\frac{df(x)}{dx} \approx \frac{f(x)-f(x-h)}{h}
	\label{eq2}
\end{equation}
as well as a difference quotient from the other side:
\begin{equation}
	\frac{df(x+h)}{dx} \approx \frac{f(x+h)-f(x)}{h}
	\label{eq3}
\end{equation}
where h is a grid spacing.
To calculate the second derivative we use formula \ref{eq3}:
\begin{equation}
	\frac{d^2f(x)}{dx^2} = \frac{df(x)}{dx} \frac{df(x)}{dx} = \frac{d}{dx} \frac{f(x+h)-f(x)}{h}
\end{equation}

\begin{equation}
	\frac{d}{dx} \frac{f(x+h)-f(x)}{h} = \frac{1}{h} \frac{d}{dx}(f(x+h)-f(x)) = \frac{1}{h} (\frac{df(x+h)}{dx} - \frac{df(x)}{dx})
\end{equation}

\noindent And thus we can substitute into \ref{eq2} and \ref{eq3}:
\begin{equation}
	\frac{d^2f(x)}{dx^2} = \frac{1}{h} (\frac{f(x+h)-f(x)}{h} - \frac{f(x)-f(x-h)}{h}) = \frac{1}{h^2} (f(x+h)-f(x)-(f(x)-f(x-h))) 
\end{equation}
\begin{equation}
	= \frac{1}{h^2} (f(x+h)-f(x)-f(x)+f(x-h)) = \frac{1}{h^2} (f(x+h)+f(x-h)-2f(x))
\end{equation}
Calculations for y and z will be similar. As long as h is the same in any dimension, we can substitute this into \ref{eq1}:
\begin{equation}
	\Delta = \frac{1}{h^2} (f(x+h)+f(x-h)-2f(x)) + \frac{1}{h^2} (f(y+h)+f(y-h)-2f(y)) + \frac{1}{h^2} (f(z+h)+f(z-h)-2f(z))
\end{equation}
\begin{equation}
	\Delta = \frac{1}{h^2} (f(x+h)+f(x-h)-2f(x) + f(y+h)+f(y-h)-2f(y) + f(z+h)+f(z-h)-2f(z))
\end{equation}
For point in space f(x,y,z), we derive 7-point stencil:

\begin{equation}
	\begin{aligned}
		\Delta = \frac{1}{h^2} \Big( 
		& f(x+h, y, z) + f(x-h, y, z) + f(x, y+h, z) \\
		& + f(x, y-h, z) + f(x, y, z+h) + f(x, y, z-h) \\
		& - 6f(x, y, z) 
		\Big)
	\end{aligned}
\end{equation}

\subsubsection{Integrating the expectation value of the potential energy operator}
\begin{equation}
	\int_{-\infty}^{+\infty} \int_{-\infty}^{+\infty} \int_{-\infty}^{+\infty} 
	x^k y^l z^m e^{-p (x^2+y^2+z^2)} \frac{1}{\sqrt{x^2+y^2+z^2}} x^k y^l z^m e^{-p(x^2+y^2+z^2)}
	dx dy dz
\end{equation}

\noindent Substitute:

\begin{align*} 
	r^2 &=  x^2+y^2+z^2 \\ 
	r &=  \sqrt{x^2+y^2+z^2} \\
	x &= r \sin{\theta} \cos{\phi} \\
	y &= r \sin{\theta} \sin{\phi} \\
	z &= r \cos{\theta} \\
	J &= r^2 \sin{\theta}
\end{align*}

\noindent Leads to:

\begin{multline}
	\int_{0}^{+\infty} \int_{0}^{2\pi} \int_{0}^{\pi} 
	(r \sin{\theta} \cos{\phi})^k
	(r \sin{\theta} \sin{\phi})^l
	(r \cos{\theta})^m 
	e^{-p r^2}
	\\
	(r \sin{\theta} \cos{\phi})^k
	(r \sin{\theta} \sin{\phi})^l
	(r \cos{\theta})^m
	e^{-p r^2} 
	\frac{1}{r}
	r^2 \sin{\theta}
	d\phi d\theta d r
\end{multline}

\noindent Reducing:

\begin{equation}
	\int_{0}^{+\infty} \int_{0}^{2\pi} \int_{0}^{\pi} 
	(r \sin{\theta} \cos{\phi})^{2k}
	(r \sin{\theta} \sin{\phi})^{2l}
	(r \cos{\theta})^{2m}
	e^{-2p r^2} 
	r \sin{\theta}
	d\phi d\theta d r
\end{equation}

\noindent Split the integrand into three functions:

\begin{equation}
	(r \sin{\theta} \cos{\phi})^{2k}
	(r \sin{\theta} \sin{\phi})^{2l}
	(r \cos{\theta})^{2m}
	e^{-2p r^2} 
	r \sin{\theta}
\end{equation}

\begin{equation}
	r^{2k} \sin^{2k}{\theta} \cos^{2k}{\phi} 
	r^{2l} \sin^{2l}{\theta}  \sin^{2l}{\phi} 
	r^{2m} \cos^{2m}{\theta} 
	e^{-2p r^2} 
	r \sin{\theta}
\end{equation}

\begin{equation}
	(r^{2k} r^{2l} r^{2m} e^{-2p r^2} r)
	(\sin^{2k}{\theta}\sin^{2l}{\theta} \cos^{2m}{\theta}  \sin{\theta} )
	(\cos^{2k}{\phi}   \sin^{2l}{\phi} )
\end{equation}

\begin{equation}
	(r^{2(k+l+m)+1} e^{-2p r^2})
	(\sin^{2(k+l)+1}{\theta}\cos^{2m}{\theta})
	(\cos^{2k}{\phi}   \sin^{2l}{\phi} )
\end{equation}
From which derives:
\begin{equation}
	\int_{0}^{+\infty} r^{2(k+l+m)+1} e^{-2p r^2} d r
	\int_{0}^{2\pi} \sin^{2(k+l)+1}{\theta}\cos^{2m}{\theta} d\theta
	\int_{0}^{\pi} \cos^{2k}{\phi}   \sin^{2l}{\phi} d\phi
\end{equation}
Which using Mathematica is calculated as equal to:
\begin{multline}
	\frac{\Gamma \left(k+\frac{1}{2}\right) \Gamma \left(l+\frac{1}{2}\right) \Gamma \left(m+\frac{1}{2}\right) 2^{-k-l-m-2} p^{-k-l-m-1} \Gamma (k+l+m+1)}{\Gamma\left(k+l+m+\frac{3}{2}\right)} \\
\end{multline}

\subsubsection{Integrating the expectation value of the kinetic energy operator}
\begin{equation}
	\int_{-\infty}^{+\infty} \int_{-\infty}^{+\infty} \int_{-\infty}^{+\infty} 
	x^k y^l z^m e^{-p (x^2+y^2+z^2)} \Delta x^k y^l z^m e^{-p(x^2+y^2+z^2)}
	dx dy dz
\end{equation}

\noindent Simplify the integrand:
\begin{equation}
	x^k y^l z^m e^{-p (x^2+y^2+z^2)} \Delta x^k y^l z^m e^{-p(x^2+y^2+z^2)}
\end{equation}

\begin{equation}
	(\frac{\partial^2}{\partial x^2} + \frac{\partial^2}{\partial y^2} + \frac{\partial^2}{\partial z^2})x^k y^l z^m e^{-p(x^2+y^2+z^2)}
\end{equation}

\noindent Split the integrand into three functions:
\begin{equation}
	\frac{\partial^2}{\partial x^2} x^k y^l z^m e^{-p(x^2+y^2+z^2)}
\end{equation}

\begin{equation}
	\frac{\partial^2}{\partial y^2} x^k y^l z^m e^{-p(x^2+y^2+z^2)}
\end{equation}

\begin{equation}
	\frac{\partial^2}{\partial z^2} x^k y^l z^m e^{-p(x^2+y^2+z^2)}
\end{equation}

\noindent Next calculate first second-order partial derivative:

\begin{equation}
	\frac{\partial^2}{\partial x} x^k y^l z^m e^{-p(x^2+y^2+z^2)}
\end{equation}

\begin{equation}
	y^l z^m \frac{\partial^2}{\partial x^2} x^k e^{-p(x^2+y^2+z^2)}
\end{equation}

\begin{equation}
	y^l z^m \frac{\partial^2}{\partial x^2} x^k e^{-px^2-py^2-pz^2}
\end{equation}

\begin{equation}
	y^l z^m \frac{\partial^2}{\partial x^2} x^k e^{-px^2}e^{-py^2}e^{-pz^2}
\end{equation}

\begin{equation}
	y^l z^m e^{-py^2}e^{-pz^2} \frac{\partial^2}{\partial x} x^k e^{-px^2}
\end{equation}

\noindent Which gives the value of first partial derivative:
\begin{equation}
	\frac{\partial^2\phi(x,y,z)}{\partial x^2} = y^l z^m e^{-py^2}e^{-pz^2}  \left(x^k (4 p^2 x^2 e^{-p x^2}-2 p e^{-p x^2})+(k-1) k x^{k-2} e^{-p x^2}-4 k p x^k e^{-p x^2}\right)
\end{equation}

\noindent Similarly, for the second mixed partial derivative, yields:

\begin{equation}
	x^k z^m e^{-px^2}e^{-pz^2} \frac{\partial^2}{\partial y^2} y^l e^{-py^2}
\end{equation}

\noindent Which gives the value of second partial derivative:
\begin{equation}
	\frac{\partial^2\psi(y)}{\partial y^2} =  x^k z^m e^{-px^2}e^{-pz^2} \left( z^m (4 p^2 z^2 e^{-p z^2}-2 p e^{-p z^2})+(m-1) m z^{m-2} e^{-p z^2}-4 m p z^m e^{-p z^2} \right)
\end{equation}

\noindent And the third yields:
\begin{equation}
	x^k y^l e^{-px^2}e^{-py^2} \frac{\partial^2}{\partial z^2} z^m e^{-pz^2}
\end{equation}

\noindent The result obtained with Mathematica:
\begin{equation}
	\frac{\partial^2\psi(z)}{\partial z^2} = x^k y^l e^{-px^2}e^{-py^2} \left( y^l (4 p^2 y^2 e^{-p y^2}-2 p e^{-p y^2})+(l-1) l y^{l-2} e^{-p y^2}-4 l p y^l e^{-p y^2}\right)
\end{equation}

\noindent Now the integral can be split into a sum of three integrals:

\begin{equation}
	I_1 = \psi \frac{\partial^2\psi(x,y,z)}{\partial x^2}
\end{equation}
\begin{equation}
	I_2 = \psi \frac{\partial^2\psi(x,y,z)}{\partial y^2}
\end{equation}
\begin{equation}
	I_3 = \psi \frac{\partial^2\psi(x,y,z)}{\partial z^2}
\end{equation}

Każda z nich była obliczana w analogiczny sposób:

\begin{equation}
	I_1 = \psi x^k y^l z^m e^{-2p(x^2+y^2+z^2)} y^l z^m e^{-py^2}e^{-pz^2} \frac{\partial^2\psi(x)}{\partial x^2}
\end{equation}

\begin{equation}
	I_1 = x^k y^l z^m e^{-2p(x^2+y^2+z^2)} \frac{\partial^2\psi(x)}{\partial x^2}
\end{equation}

\fbox{$2^{-4 k-2 m-11} \Gamma \left(k-\frac{1}{2}\right) p^{-2 k-m-\frac{1}{2}} \left((1-4 k) \Gamma \left(k+\frac{1}{2}\right) \Gamma \left(m+\frac{1}{2}\right)-4^{k-l} (k (8 m-3)-3 m+1) \Gamma \left(l+\frac{1}{2}\right) \Gamma \left(m-\frac{1}{2}\right) p^{k-l}\right)\text{ if }\Re(k)>\frac{1}{2}\land l^*+l>-1\land \Re(m)>\frac{1}{2}\land \Re(p)>0$}


\subsubsection{Optimization of the normalization factor}
The GTO basis function is equal:
\begin{equation}
	\text{GTO} = x^k y^l z^m e^{(-p(x^2 + y^2 + z^2))}
\end{equation}
Specific case: $k = l = m = 0$
\begin{equation}
	\text{GTO}_0 = e^{(-p(x^2 + y^2 + z^2))}
\end{equation}
Substituting
\begin{equation}
	\text{GTO}_{0r} = e^{(-pr^2)}
\end{equation}
Multiplying times the Jacobian
\begin{equation}
	\text{GTO}_{0rj} = \exp(-pr^2) r^2 \sin(\theta)
\end{equation}
Separating radial part
\begin{equation}
	\text{GTO}_{0rj,\text{radial}} = \exp(-pr^2) r^2
\end{equation}
Computing integral of the radial part
\begin{equation}
	\int_0^\infty \frac{\exp(-pr^2) r^2}{r} \, dr = \frac{1}{4p}
\end{equation}
Computing integral of the sine of \Theta
\begin{equation}
	\int_0^\pi \sin(\theta) \, d\theta \int_0^{2\pi} d\phi = 4\pi
\end{equation}
Getting potential part of the expected value of the Hamiltonian
\begin{equation}
	\text{Potential Integral} = - \frac{\pi}{p}
\end{equation}
Some auxiliary computations
\begin{equation}
	\text{Kinetic Integrals} = 
	\left\{
	-\sqrt{p} \sqrt{\frac{\pi}{2}}, 
	-\sqrt{p} \sqrt{\frac{\pi}{2}}, 
	-\sqrt{p} \sqrt{\frac{\pi}{2}}
	\right\}
\end{equation}

\begin{equation}
	\text{Kinetic Energy Integral} = 
	\frac{3 \pi^{3/2}}{4 \sqrt{2} \sqrt{p}}
\end{equation}

\begin{equation}
	\text{Norm Integral} = \frac{\pi^{3/2}}{2 \sqrt{2} p^{3/2}}
\end{equation}

\begin{equation}
	\text{Rayleigh Quotient} = \frac{\text{Kinetic Integral} + \text{Potential Integral}}{\text{Norm Integral}}
\end{equation}

\begin{equation}
	\text{Minimizer of } p: p = \frac{8}{9\pi}
\end{equation}

\begin{equation}
	\text{Minimal Rayleigh Quotient Value: } -\frac{4}{3\pi}
\end{equation}

\begin{equation}
	\text{Potential Part of Rayleigh Quotient: } -2\sqrt{p} \sqrt{\frac{2}{\pi}}
\end{equation}

\begin{equation}
	\text{Kinetic Part of Rayleigh Quotient: } \frac{3 \pi^{3/2}}{4 \sqrt{2} \sqrt{p}}
\end{equation}
